
\documentclass[a4paper,12pt]{article} % тип документа

% report, book

%  Русский язык

\usepackage[T2A]{fontenc}			% кодировка
\usepackage[utf8]{inputenc}			% кодировка исходного текста
\usepackage[english,russian]{babel}	% локализация и переносы


% Математика
\usepackage{amsmath,amsfonts,amssymb,amsthm,mathtools} 

%форматирование
\usepackage{float}
\usepackage{geometry}
\geometry{top    = 5mm,
		  left   = 5mm,
		  right  = 5mm,
		  bottom = 5mm}
		  
%Рисунки
\usepackage{graphicx}
\usepackage{wrapfig}

\usepackage{hyperref}
\usepackage[rgb]{xcolor}
\hypersetup{
	colorlinks = true,
	urlcolor = blue
}

\usepackage{wasysym}

%Заговолок
\author{Колобаев Дмитрий\\Группа Б01-903}
\title{Лампа с дистанционным выключателем.} % Номер и название работы


\begin{document}
\maketitle

\section*{Описание}
Проект включает в себя два независимых устройства - пульт и лампу. При нажатии кнопки на пульте, генерируется сигнал ИК диапозона саециальной формы, которая может быть воспринята датчиком стоящим на лампе. Когда датчик фиксирует сигнал от пульта, меняется состояние диода-лампы. Таким образом можно управлять на лампой на расстоянии с помощью пульта.

\section*{Схема}
\begin{figure}[h]
\begin{minipage}[b]{0.45\textwidth}
\includegraphics[width=\textwidth]{remote.png}
\center{Схема пульта.}
\end{minipage}
\begin{minipage}[b]{0.45\textwidth}
\includegraphics[width=\textwidth]{lamp.png}
\center{Схема лампы.}
\end{minipage}
\end{figure}

\newpage

\section*{Программа}

\subsection*{Программа для пульта управления.}

\begin{verbatim}
; ATtiny 2313
; Remote controller
.equ DDRD = 0x11
.equ PIND = 0x10
.equ PORTD = 0x12
.equ DDRB = 0x17
.equ PORTB = 0x18
.equ PINB = 0x16
.equ GIMSK = 0x3b
.equ MCUCR = 0x35
.equ TCCR0A = 0x30
.equ TCCR0B = 0x33
.equ OCR0A = 0x36
.equ TIMSK = 0x39
.equ CLKPR = 0x26
.CSEG
rjmp RESET
rjmp INT0
nop
nop
nop
nop
nop ;overflow
nop
nop
nop
nop
nop
nop
rjmp T0COMPA; cmp match

; r15 - 0
; r16 - temporary
; r17 - FF
; r18 - timer out/cmp value
; r19 - TIMSK state
; r20 - package counter
; Output signal goes to port B

RESET:
        cli
        ser r17
        clr r15
        ; clock setup
        ldi r16, 128
        out CLKPR, r16
        out CLKPR, r15
        ; setup MCUCR 0010--10
        ldi r16, 0x22
        out MCUCR, r16
        ;setup GIMSK
        ldi r16, 64
        out GIMSK, r16
        ;setup ports
        out DDRB, r17
        out PORTB,r15
        out DDRD, r15
        out PORTD, r17
        ;setup timer
        ldi r16, 2
        out TCCR0A, r16
        out TCCR0B, r15
        ldi r16, 1
        out TIMSK, r16
        clr r20
        ; OCR0A setup
        clr r18
        out OCR0A, r18
        ldi r16, 1
        out TCCR0B, r16
        sei
        rjmp wait

wait:
        rjmp wait

T0COMPA:
        cli
        cpi r18, 20
        ldi r18, 20
        ser r16
        brne noaction
        ldi r18, 123
        clr r16
        inc r20
noaction:
        out OCR0A, r18
        out PORTB, r16
        cpi r20, 10
        brne return
        clr r20
        out TCCR0B, r15
return:
        sei
        reti

INT0:
        cli
        clr r20
        ; OCR0A setup
        clr r18
        out OCR0A, r18
        ldi r16, 1
        out TCCR0B, r16
        sei
        reti
\end{verbatim}

\newpage

\subsection*{Программа для лампы.}

\begin{verbatim}
; ATtiny13A
.equ DDRB = 0x17
.equ PINB = 0x16
.equ PORTB = 0x18
.equ GIMSK = 0x3b
.equ MCUCR = 0x35
.equ SPL = 0x3d
.CSEG
rjmp reset
rjmp int0

reset:
	cli
	sbi DDRB, 4
	ldi r16, 0x2D
	out PORTB, r16
	; setup MCUCR
	ldi r16, 3
	out MCUCR, r16
	;setup GIMSK
	ldi r16, 64
	out GIMSK, r16
	sei
	rjmp wait

wait:
	rjmp wait


int0:
	cli
	; toggle pin
	sbi PINB, 4
	sei
	reti
\end{verbatim}

\end{document}